\documentclass[12pt,a4paper]{article}
\usepackage{amssymb}
\usepackage{amsmath}


%\usepackage[ngerman]{babel}    %Trennungen, Schriftsatz; Neue deutsche Rechtschreibung
\usepackage[T1]{fontenc}       %Umlaute, Sonderzeichen...
\usepackage[utf8]{inputenc}

\usepackage[top=3cm, bottom=3cm, left=2cm, right=2cm]{geometry}

\begin{document}
\thispagestyle{empty} 
\begin{center}
{\large The lower central series of the unit group of an integral group ring}\\
\vspace*{.5cm}
Sugandha Maheshwary\\
Indian Institute of Technology Roorkee\\
\end{center}
\vspace*{.8cm}

{\bf Abstract:} For a group $G$, denote by $\mathcal{V}(\mathbb{Z} G)$, the group of normalized units, i.e., units with augmentation one in the integral group ring $\mathbb{Z} G$. The study of $\mathcal{V}(\mathbb{Z} G)$ and its center attracts a varied set of questions and one naturally seeks the understanding of central series of $\mathcal{V}(\mathbb{Z} G)$. While the upper central series of $\mathcal{V}(\mathbb{Z} G)$ has been well explored, at least for a finite group G, apparently, not much is known about its lower central series $\{ \gamma_n(\mathcal{V}) \}_{n > 1}$ where $\mathcal{V} := \mathcal{V}(\mathbb{Z}G)$ and 
$$
\gamma_{1}(\mathcal{V})=\mathcal{V}, \gamma_{2}(\mathcal{V})=\mathcal{V}^{\prime}, \gamma_{i}(\mathcal{V})=\left[\gamma_{i-1}(\mathcal{V}), \mathcal{V}\right], i \geq 2$$

In this talk, I will try to draw attention towards certain fundamental problems associated to the study of the lower central series of $\mathcal{V}(\mathbb{Z} G)$ and present some recent advancements. In particular, I will present some results on the abelianisation of the $\mathcal{V}(\mathbb{Z} G)$. I would also like to discuss a natural filtration of the unit group $\mathcal{V}(\mathbb{Z} G)$ analogous to the filtration of the group $G$ given by its dimension series, leading to results on residual nilpotence of $\mathcal{V}(\mathbb{Z} G)$.		


\end{document}