\documentclass[12pt,a4paper]{article}
\usepackage{amssymb}
\usepackage{amsmath}
\usepackage{enumitem}

%\usepackage[ngerman]{babel}    %Trennungen, Schriftsatz; Neue deutsche Rechtschreibung
\usepackage[T1]{fontenc}       %Umlaute, Sonderzeichen...
\usepackage[utf8]{inputenc}

\usepackage[top=3cm, bottom=3cm, left=2cm, right=2cm]{geometry}

\begin{document}
\thispagestyle{empty} 
\begin{center}
{\large Continuum Braid group}\\
\vspace*{.5cm}
Margherita Paolini\\
Sapienza University of Rome\\
\end{center}
\vspace*{.8cm}

{\bf Abstract:} In the foundational manuscript [1] Emil Artin has introduced the
sequence of Braid Group $B_n$. $B_n$ is a group whose elements are equivalence classes of $n$-braids up to isotopy. The Braid Group admits different equivalent definitions, in particular, we will introduce the Birman-Ko-Lee presentation [2] whose generators are $a_{l,m}$ (the $a_{l,m}$ braid is the elementary interchange of the
$l$-th and the $m$-th strand of the braid with all the other strands held fixed). A classical result done by Lusztig [3] shows that there exists an action of the Braid Group over the Drinfel-Jimbo Quantum group ($U_q^{DJ}$); this action plays a central role in order to understand the structure of $U_q^{DJ}$. In recent years Appel, Sala and Schiffmann [3], [4] introduced a continuum analogue Quantum Group $U_q^{DJ}(X)$, that is an appropriate colimit of DJ Quantum Groups and their Cartan datum $X$ can be thought of as a generalization of a quiver, where vertices are replaced by intervals. In order to study these continuum Quantum Groups, we define a continuum analogue of Braid Groups $B_X$ mean by the BKL generators. We show that these groups preserve the colimit structure, we show that the Theorem of Hiwahori and Matsumoto holds [6] for the BKL presentation of $B_n$ and it is compatible with the colimit structure.

\bigskip
\textsc{References}

\begin{enumerate}[label={[\arabic*]}]
\item E. Artin, Theorie der Z\"opfe, Amburg Abh. 4, (1925), 47-72
\item J. Birman, K.H. Ko and S.J. Lee, A new approach to the word and conjugacy
problem in the braid group, Adv. Math., 139 (2), (1998), 322-253
\item  G. Lusztig, “Introduction to Quantum Groups”, Birk\"auser, 1993
\item A. Appel and F. Sala, Quantization of Continuum Kac Moody Algebras, Pure Appl. Math. Q. 16, (2020), 439-493
\item A. Appel, F. Sala and O. Schiffman, Continuum Kac-Moody algebras, Moscow Mathematical Journal 22 (2022), 48pp
\item N. Iwahori and H. Matsumoto, On some Bruhat decomposition and the structure of the Hecke rings of $p$-adic Chevalley groups, Inst. Hautes Etudes Sci. Publ. Math. (1965), no. 25, 5–48
\end{enumerate}


\end{document}