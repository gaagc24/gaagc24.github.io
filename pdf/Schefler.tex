\documentclass[12pt,a4paper]{article}
\usepackage{amssymb}
\usepackage{amsmath}


%\usepackage[ngerman]{babel}    %Trennungen, Schriftsatz; Neue deutsche Rechtschreibung
\usepackage[T1]{fontenc}       %Umlaute, Sonderzeichen...
\usepackage[utf8]{inputenc}

\usepackage[top=3cm, bottom=3cm, left=2cm, right=2cm]{geometry}

\begin{document}
\thispagestyle{empty} 
\begin{center}
{\large  Seperating Noether number of finite abelian groups}\\
\vspace*{.5cm}
Barna Schefler\\
Eötvös Loránd University\\
\end{center}
\vspace*{.8cm}

{\bf Abstract:} The separating Noether number $\beta_{sep}(G)$ of a finite group $G$ is the minimal positive integer $d$ such that for any finite dimensional complex representation of $G$, the homogeneous polynomial $G$-invariants of degree at most $d$ form a separating set. This is modeled on the Noether number $\beta(G)$. For a finite abelian group $G$, we have $\beta(G) = D(G)$, where $D(G)$, the Davenport constant of $G$ is the maximal length of an irreducible zero-sum sequence over $G$. An open question concerning the Davenport constant is whether the equality $D(C_n^r) = 1 + r(n-1)$ (or $\beta(C_n^r) = 1 + r(n-1)$) holds (where $C_n^r$ stands for the
direct sum of $r$ copies of the cyclic group of order $n$). The analogous question on the separating Noether will be a main point of this presentation. The most common families of finite abelian groups for which the exact value of the Davenport constant is known are the finite abelian groups of rank two and the finite abelian $p$-groups. Hence it is natural to compute $\beta_{sep}(G)$ for rank two abelian
groups. This case will also appear in our talk.

											


\end{document}