\documentclass[12pt,a4paper]{article}
\usepackage{amssymb}
\usepackage{amsmath}


%\usepackage[ngerman]{babel}    %Trennungen, Schriftsatz; Neue deutsche Rechtschreibung
\usepackage[T1]{fontenc}       %Umlaute, Sonderzeichen...
\usepackage[utf8]{inputenc}

\usepackage[top=3cm, bottom=3cm, left=2cm, right=2cm]{geometry}

\begin{document}
\thispagestyle{empty} 
\begin{center}
{\large  Component groups of the stabilizers of nilpotent orbit representatives}\\
\vspace*{.5cm}
Emanuele Di Bella\\
University of Trento\\
\end{center}
\vspace*{.8cm}

{\bf Abstract:} The theory of nilpotent orbits of simple Lie algebras has seen tremendous developments over the past decades. In this context an important role is played by the component group of the stabilizer of a nilpotent element. In this talk, the aim is to show computational methods to obtain explicit generators of the component group of the centralizer of a nilpotent element in a simple Lie algebra over $\mathbb{C}$. In some cases such generators had already been determined by impressive hand calculations but these often use the specific form of the chosen representative and are thus not immediately applicable to different representatives of the same orbit. It is then interesting to show how to overcome this issue constructing specific algorithms: for the classical types there is a straightforward method that directly translates well-known theoretical constructions; for the exceptional types we devise a method using the double centralizer of an $\mathfrak{sl}_2$-triple. In particular, it gives an independent construction of the component group, that does not depend on the prior knowledge of its isomorphism type.


For our purposes, we needed to construct many algorithms which have been mainly implemented in the computational algebra system GAP.


\end{document}