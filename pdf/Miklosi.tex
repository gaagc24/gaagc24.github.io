\documentclass[12pt,a4paper]{article}
\usepackage{amssymb}
\usepackage{amsmath}


%\usepackage[ngerman]{babel}    %Trennungen, Schriftsatz; Neue deutsche Rechtschreibung
\usepackage[T1]{fontenc}       %Umlaute, Sonderzeichen...
\usepackage[utf8]{inputenc}

\usepackage[top=3cm, bottom=3cm, left=2cm, right=2cm]{geometry}

\begin{document}
\thispagestyle{empty} 
\begin{center}
{\large  Symmetric polynomials over finite fields}\\
\vspace*{.5cm}
Botond Mikl\'osi\\
Eötvös Loránd University\\
\end{center}
\vspace*{.8cm}

{\bf Abstract:} Consider the action of the symmetric group $S_n$ on the $n$-dimensional vector space over the finite field $\mathbb{F}_q$ of $q$ elements, where $q$ stands for a prime power $p^k$. We have an induced action on $\mathbb{F}_q[x_1, \ldots, x_n]$  the coordinate ring of $\mathbb{F}_q^n$. Kemper, Lopatin and Reimers proved that the elementary symmetric polynomials of degree $2k$ form a separating set of minimal size in the invariant ring over the $2-$element finite field. Based on their paper we have managed to exploit this result: over an arbitrary finite field $\mathbb{F}_q$ the set of elementary symmetric polynomials of degree $j p^k$ (with $j \in \{ 0, \ldots, q-1\}, \ k \in \mathbb{Z}_{> 0}$ and $j p^k$ smaller or equal to $n$) form a separating set. Moreover, this separating set is not far from being minimal when $q = p$ and the dimension is large compared to $p$. In the talk I will present the main ideas and the outline of the proof.


\end{document}