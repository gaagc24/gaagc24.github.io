\documentclass[12pt,a4paper]{article}
\usepackage{amssymb}
\usepackage{amsmath}


%\usepackage[ngerman]{babel}    %Trennungen, Schriftsatz; Neue deutsche Rechtschreibung
\usepackage[T1]{fontenc}       %Umlaute, Sonderzeichen...
\usepackage[utf8]{inputenc}

\usepackage[top=3cm, bottom=3cm, left=2cm, right=2cm]{geometry}

\begin{document}
\thispagestyle{empty} 
\begin{center}
{\large  Chiralisation of reduction by stages}\\
\vspace*{.5cm}
Thibault Juillard\\
University of Paris-Saclay\\
\end{center}
\vspace*{.8cm}

{\bf Abstract:} The dual space of a simple Lie algebra is a well-known example of a Poisson variety. This variety admits a ``chiralisation'' (i.e. a quantisation by a vertex algebra) by some Kac-Moody vertex algebra. Given a nilpotent element in the simple Lie algebra, using Hamiltonian reduction, one can construct a new Poisson variety: the Slodowy slice associated to this nilpotent element. This variety is itself chiralised by an affine W-algebra, which is the BRST reduction of the Kac-Moody vertex algebra. Indeed, this BRST reduction is a cohomological analogue of the Hamiltonian reduction. 

I will present a joint work with Naoki Genra about the problem of reduction by stages of Slodowy slices (arXiv:2212.06022) and a more recent work about the analogue problem for affine W-algebras. Given a suitable pair of nilpotent elements in the Lie algebra, it is possible to reconstruct one of the two Slodowy slices (resp. affine W-algebras) as the reduction of the other slice (resp. algebra). As an application, I will explain how this reduction by stages allows us to prove vertex algebra analogues of classical results coming from the study of nilpotent singularities by Kraft and Procesi in the 80's.


\end{document}