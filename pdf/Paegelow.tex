\documentclass[12pt,a4paper]{article}
\usepackage{amssymb}
\usepackage{amsmath}


%\usepackage[ngerman]{babel}    %Trennungen, Schriftsatz; Neue deutsche Rechtschreibung
\usepackage[T1]{fontenc}       %Umlaute, Sonderzeichen...
\usepackage[utf8]{inputenc}

\usepackage[top=3cm, bottom=3cm, left=2cm, right=2cm]{geometry}

\begin{document}
\thispagestyle{empty} 
\begin{center}
{\large  The Procesi bundle over the ${\displaystyle \Gamma }$-fixed points of the punctual Hilbert scheme in $\mathbb{C}^2$}\\
\vspace*{.5cm}
Raphaël Paegelow\\
University of Montpellier\\
\end{center}
\vspace*{.8cm}

{\bf Abstract:} During my thesis, which I have done with C\'edric Bonnaf\'e and that I will defend on June 19, I have studied the fixed point locus of the punctual Hilbert scheme on $\mathbb{C}^2$ under the group action of the finite subgroups of $SL_2(\mathbb{C})$. This has been done using quiver varieties on the McKay quiver attached to the finite subgroups. Let $\Gamma$ be a finite subgroup of $SL_2(\mathbb{C})$ and $S_n$ be the symmetric group on $n$ letters. Denote by $\Gamma_n$ the wreath product of $S_n$ with $\Gamma$. In this setting, I have shown that one can also retrieve all the projective and symplectic resolutions of the singularity $(\mathbb{C}^2)^n/\Gamma_n$, classified by Gwyn Bellamy and Alastair Craw, as irreducible components of the $\Gamma$-fixed points of $k$ points in $\mathbb{C}^2$ where $k$ depends on the resolution. Moreover, the indexing set of the irreducible components of the $\Gamma$-fixed point locus leads to interesting combinatorics in type $A$ and $D$ in terms of cores of partitions. Another direction has been to study the Procesi bundle over these irreducible components. To be more precise, I have studied the action of the group $S_n \times \Gamma$ on the fibers of the Procesi bundle over the $\Gamma$-fixed points of the Hilbert scheme of $n$ points in $\mathbb{C}^2$. A joint work with Gwyn Bellamy using the geometry of the Procesi bundle and the geometry of the isospectral Hilbert scheme can be interpreted as a reduction to ''cuspidal'' fibers of the Procesi bundle. A first conjecture into the study of these cuspidal fibers seems to be using the Fock representation of the affine Kac-Moody algebra attached to $\Gamma$. 
%My work, which is at the intersection of combinatorics, representation theory and algebraic geometry makes me very interested in the workshop that you are organizing.

\end{document}