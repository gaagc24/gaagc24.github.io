\documentclass[12pt,a4paper]{article}
\usepackage{amssymb}
\usepackage{amsmath}


%\usepackage[ngerman]{babel}    %Trennungen, Schriftsatz; Neue deutsche Rechtschreibung
\usepackage[T1]{fontenc}       %Umlaute, Sonderzeichen...
\usepackage[utf8]{inputenc}

\usepackage[top=3cm, bottom=3cm, left=2cm, right=2cm]{geometry}

\begin{document}
\thispagestyle{empty} 
\begin{center}
{\large  Division and Localization on Groupoid Graded Rings}\\
\vspace*{.5cm}
Caio Antony\\
University of São Paulo\\
\end{center}
\vspace*{.8cm}

{\bf Abstract:} A groupoid is a small category in which every morphism is invertible, and as such, generalizes the concept of groups. The concept of a groupoid graded ring is similar to that of group graded ring. That is, there exists additive subgroups for each arrow in the groupoid, whose multiplication makes sense with the composition law of the arrows if they can be composed, or is equal to 0 otherwise. Different from group graded rings, groupoid graded rings do not need to be unital. We suppose that our graded rings are object unital, that is, for every object in the groupoid, there exists an idempotent in the ring which acts as unity for products with homogeneous elements of compatible degrees.

Division is studied with respect to the aforementioned idempotents. In this talk, we'll discuss recent progress regarding division and localization on object unital groupoid graded rings. In particular, we'll discuss a generalization of P.M. Cohn's results which characterize homomorphisms from a ring to a division ring, previously generalized by D. E. N. Kawai and J. Sanchez to the context of group graded rings.

This research has been developed under the supervision of professors Javier Sánchez, from Universidade de São Paulo, and Ángel del Río, from Universidad de Murcia, and has been supported by FAPESP grants 2022/11166-6 and 2023/11994-9.


\end{document}