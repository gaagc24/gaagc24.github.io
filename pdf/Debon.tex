\documentclass[12pt,a4paper]{article}
\usepackage{amssymb}
\usepackage{amsmath}
\usepackage{enumitem}


%\usepackage[ngerman]{babel}    %Trennungen, Schriftsatz; Neue deutsche Rechtschreibung
\usepackage[T1]{fontenc}       %Umlaute, Sonderzeichen...
\usepackage[utf8]{inputenc}

\usepackage[top=3cm, bottom=3cm, left=2cm, right=2cm]{geometry}

\begin{document}
\thispagestyle{empty} 
\begin{center}
{\large  The classification of the prime graphs of finite solvable rational/cut groups}\\
\vspace*{.5cm}
Sara C. Deb\'on\\
University of Murcia\\
\end{center}
\vspace*{.8cm}

{\bf Abstract:} Graphs defined on groups constitute a powerful tool in the study of finite groups. The object of interest is to understand which -and up to which extend- graph-theoretical notions can be translated to group properties. Examples of graphs associated to groups are the commuting graph, the cyclic graph or the prime graph. An exposition of different graphs defined on groups can be found in [2]. 
	Let $G$ be a finite group. The prime graph or Gruenberg-Kegel graph of $G$ is the undirected graph whose vertices are the primes dividing the order of $G$ and an edge connects a pair of different vertices $p$ and $q$ if and only if $G$ contains an element of order $pq$. The prime graph reflects interesting properties of the base group, for instance, a graph is isomorphic to the prime graph of a finite solvable group if and only if its complement is 3-colorable and triangle-free [3]. Due to this result, several mathematicians have been dedicated to the study of the prime graphs of some classes of solvable groups. In particular, we are interested in the prime graphs of finite solvable groups which are cut or rational.  
	A group $G$ is cut if for every $g$ in $G$ each generator of $\langle g \rangle$ is conjugate to $g$ or $g^{-1}$. A group $G$ is rational if for every $g \in G$ all generators of $\langle g \rangle$ are conjugate. The classification of the prime graphs of finite solvable cut/rational groups was initiated in [1]. A graph is left to complete the classification in the rational case and four in the cut case. The aim of this talk is to share recent advances in the classification and to discuss what is remaining.


\bigskip
\textsc{References}

\begin{enumerate}[label={[\arabic*]}]
\item A. Bächle, A. Kiefer, S. Maheshwary, and \'A. del Río, 'Gruenberg–Kegel graphs: cut groups, rational groups and the prime graph question', Forum Mathematicum 35 (2023), no. 2, pp. 409–429, doi:10.1515/forum-2022-0086.
\item P.J. Cameron, 'Graphs defined on groups', International Journal of Group Theory 11 (2022), no.2, pp.53-107, doi: 10.22108/ijgt.2021.127679.1681.
\item A. Gruber, T. M. Keller, M. L. Lewis, K. Naughton, and B. Strasser, 'A characterization of the prime graphs of solvable groups', J. Algebra 442 (2015), pp.397–422.
\end{enumerate}



\end{document}