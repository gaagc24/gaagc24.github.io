\documentclass[12pt,a4paper]{article}
\usepackage{amssymb}
\usepackage{amsmath}


%\usepackage[ngerman]{babel}    %Trennungen, Schriftsatz; Neue deutsche Rechtschreibung
\usepackage[T1]{fontenc}       %Umlaute, Sonderzeichen...
\usepackage[utf8]{inputenc}

\usepackage[top=3cm, bottom=3cm, left=2cm, right=2cm]{geometry}

\begin{document}
\thispagestyle{empty} 
\begin{center}
{\large   Affine Bruhat order and Kazhdan-Lusztig polynomials for $p$-adic Kac-Moody groups}\\
\vspace*{.5cm}
Paul Philippe\\
Institut Camille Jordan\\
\end{center}
\vspace*{.8cm}

{\bf Abstract:} The Iwahori-Hecke algebra is a crucial tool in the study of a reductive group over local fields, it admits a basis indexed by the associated affine Weyl group. In the general Kac-Moody setting, an equivalent was constructed by N. Bardy-Panse, S. Gaussent and G. Rousseau in 2016, defined by generators and relations over a basis indexed by a semi-group $W^+$ which plays the role of the affine Weyl group. Unlike in the reductive case, $W^+$ is no longer a (extended) Coxeter group, which makes classical Kazhdan-Lusztig theory inapplicable in this context. However in 2018 D. Muthiah and D.Orr have managed to define an order and a length function on $W^+$ analogous to the Bruhat order and the Bruhat length. Moreover, Muthiah gave a strategy to define Kazhdan-Lusztig polynomials for
these algebras, using measures. We present several properties recently obtaine on this $W^+$-order, and their implications. This is a joint work with A. H\'ebert.


\end{document}