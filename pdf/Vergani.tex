\documentclass[12pt,a4paper]{article}
\usepackage{amssymb}
\usepackage{amsmath}
\usepackage{enumitem}


%\usepackage[ngerman]{babel}    %Trennungen, Schriftsatz; Neue deutsche Rechtschreibung
\usepackage[T1]{fontenc}       %Umlaute, Sonderzeichen...
\usepackage[utf8]{inputenc}

\usepackage[top=3cm, bottom=3cm, left=2cm, right=2cm]{geometry}

\begin{document}
\thispagestyle{empty} 
\begin{center}
{\large  Rationality of finite groups: Groups with quadratic field of values}\\
\vspace*{.5cm}
Marco Vergani\\
University of Florence\\
\end{center}
\vspace*{.8cm}

{\bf Abstract:} The main topic of this talk is families of groups that have a characterization of their integral central units inside the rational group algebra. Using representation theory it is possible to consider groups as acting over vector spaces in a natural way, relating the
irreducible actions to the field generated by the trace of the representation. Those fields give us a lot of information about the group itself. In this talk we will focus on groups with
field of values that are quadratic extensions of the rationals and we will define tools that allow us to detect how far the group is from a ''rational'' action.

\bigskip
\textsc{References}

\begin{enumerate}[label={[\arabic*]}]
\item Seyed Hassan Alavi, Ashraf Daneshkhah, and Mohammad Reza Darafsheh. “On semi-rational Frobenius groups”. In: Journal of Algebra and its applications 15.2 (2015)
\item Sugandha Maheshwary Andreas B\"achle Ann Kiefer and \'Angel del R\'io. ''Gruenberg-Kegel graphs: cut groups, rational groups and the prime graph question''. In: (2022).
\item Andreas B\"achle.''Integral Group Rings of Solvable Groups with Trivial Central Units''. In: (2017).
\item Andreas B\"achle et al. ''Global and local proprieties of finite groups with only finitely many central units in their integral group ring''. In: (2021).
\item David Chillag and Silvio Dolfi. ''Semi-rational solvable groups.'' In: J. Group Theory 13 (2010), pp. 535–548.
\item Gabriel Navarro and Joan Tent. ''Rationality and Sylow 2-subgroups''. In: Proceedings of the Edinburgh Mathematical Society 53 (2010), pp. 787–798.
\end{enumerate} 


\end{document}